It has  been over  half a century  since renowned  astrophysicist Sjur
Refsdal first hypothesized  the use of a supernova  (SN) resolved into
multiple images  as a  cosmological tool. In  recent years,  the first
multiply-imaged core-collapse (CC) SN Refsdal \citep{Kelly:2015a}, and
subsequently the first Type Ia SN iPTF16geu, have been discovered
\citep{Goobar:2016}. As the light for each of the multiple images
follows a  different path through  the expanding universe  and through
the  lensing potential,  the SN  images appear  delayed by  hours (for
galaxy-scale lenses) or years  (for cluster-scale lenses). For objects
like  SN Refsdal,  measurement of  this time  delay can  be used  as a
precise test of  cluster lens models \citep{Treu:2015}. For  a SN like
iPTF16geu, the time  delay can lead to a  \textbf{direct constraint on
the  Hubble constant  $H_0$}, that  is completely  independent of  the
local  distance  ladder.   The  next  decade  is   expected  to  yield
observations    of    tens     to    hundreds    of    multiply-imaged
SNe \citep{Oguri:2010},  yet there is  no public software  package for
analyzing multiply-imaged SNe.

\noindent\underline{\textit{Intellectual Merit}} :
As the  PI on  an ongoing  HST Archival  Research Grant,  \textbf{I am
developing the first open-source software package} in Python that will
enable a user to precisely measure lens properties and time delays for
the  hundreds  of multiply  imaged  SNe  expected in  the  LSST/WFIRST
era.  Properties of  dark  energy  and dark  matter  are still  poorly
understood  and  inadequately  constrained,  but  measurement  of  the
lensing magnification and  time delays can be used to  test models for
the      dark     matter      distribution     in      the     lensing
object  \citep{Rodney:2015a,Rodney:2016}   or  as  a  probe   to  test
cosmological models \citep{Suyu:2014}.  To that end, I  will first use
the  software  I   am  creating  to  make  more   precise  time  delay
measurements for the two currently  documented multiply imaged SNe. SN
iPTF16geu is Type  Ia, so by providing a more  accurate time delay and
luminosity  distance measurement,  I will  be able  to accomplish  two
critical  goals. First,  directly measuring  the source  magnification
will provide  an important milestone  in breaking degeneracies  in the
lens    model    \citep{Kolatt:1998,Oguri:2003b}.   Second,    precise
determinations of  the time delays  for a  multiply imaged Type  Ia SN
will provide a constraint on $H_0$, and the methodology and software I
develop will be \textbf{essential to future SN surveys} for tightening
those constraints. Between SN iPTF16geu and SN Refsdal, I will be able
to validate the  software package and methodology I  am developing for
time  delay  measurements,  and  provide  these  tools  for  the  next
generation   of  observations   with   JWST,  WFIRST,   and  LSST.   A
standardized, validated, open-source tool will need to be available to
quickly  and accurately  make  these measurements  in  the future,  as
gaining a sample of ~150 time  delay measurements can tighten the area
of uncertainty for dark energy  equation of state parameters $w_0$ and
$w_a$ by a factor of 4.8 \citep{Linder:2011}.

Observations either  by way of  gravitational lensing, or by  the next
generation telescope  JWST, will  provide a  sample of  extremely high
redshift SNe. Of  interest in the early  universe are pair-instability
(PI) and  Population III (Pop III)  SNe. The progenitors of  PISNe are
massive  stars in  low  metallicity environments  including the  early
universe,  which retain  their high  mass until  their death  as a  CC
SN. POP  III stars are  hypothesized to form around  $z\approx30$, but
have yet to be observed despite  models predicting that a POP III star
should be visible  to current telescopes through  a gravitational lens
during its death as  a CC SN \citep{marri:1998}. \textbf{Understanding
these first stars  is crucial to a wide range  of cosmology, including
the  formation of  primeval galaxies,  initial stages  of cosmological
reionization,    and     the    origin    of     Supermassive    black
holes}  \citep{Whalen:2013}.Therefore,  I  will collect  a  sample  of
current high-z  and gravitationally  lensed SNe, and  begin discerning
the properties of these first stars. In addition to searching for, and
studying the physics of PI and Pop III SNe, I will include theoretical
light curve templates  for both objects in the  open-source software I
am developing, so that future  observations can be identified and both
lens properties and progenitor physics  studied. Not all SNe found and
analyzed will be strongly lensed and valuable for identifying lens and
dark energy properties or dark  matter distribution. Therefore, I will
also  collect a  sample  of  these weakly  lensed  SNe, and  carefully
measure their magnification in order  to make small corrections in the
Hubble   diagram.  This   will  ensure   that  the   full  sample   of
gravitationally lensed SNe  are made useful, and that  the accuracy of
the current $H_0$ measurement is improved.

\

\noindent\underline{\textit{Research Outline}}:
\begin{enumerate}
\item
Complete  the python  software package  titled \textit{Supernova  Time
Delays} (SNTD).
\item
\textbf{Write publication} presenting SNTD capabilities and validation.
\item
Improve precision  on time  delay measurements for  SN Refsdal  and SN
iPTF16geu using SNTD.
\item
\textbf{Write publication} presenting new time delay measurements, and the methodology required for these analyses and obtaining a constraint for $H_0$ from iPTF16geu.
\item
Use theoretical  models to  include PISN  and POP  III SN  light curve
templates in SNTD.
\item
Obtain  gravitationally  lensed and  high-z  SN  sample to  study  the
physics  of PISNe  and POP  III SNe,  as well  as make  Hubble diagram
adjustments with weakly lensed SNe from sample.
\item
\textbf{Write publication} detailing magnification corrections for Hubble diagram SNe and their affect on $H_0$.
\item
\textbf{Write publication} detailing search for, and ideally discovery of, PISNe and POP III SNe and their physical properties.
\end{enumerate}

\noindent\underline{\textit{Broader Impacts Summary: }}
In my personal  statement, I described how my time  as a PhD candidate
at the  University of South Carolina  will have a strong  impact on my
community, but it will have a clear effect on the scientific community
as well. The SNTD software package will  be a crucial tool in years to
come, as the  next generation of telescopes  drastically increases our
catalogue of multiply  imaged SNe. Improving our  understanding of the
first stars will be essential to various branches of cosmology such as
the origin  of supermassive black holes,  and I will pave  the way for
future PISN and POP III SN observations by including their theoretical
templates in SNTD, and search for  the first POP III SN observation by
forming a sample of strongly  lensed SNe. Finally, by documenting SNTD
and my improved measurement of  the Refsdal and iPTF16geu time delays,
I will  provide a constraint  on $H_0$ and a  standardized methodology
that will be used for future SN observations to tighten constraints on
$H_0$ and dark energy properties.
