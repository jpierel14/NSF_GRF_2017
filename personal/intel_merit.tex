Growing up, I had an  intense passion for astronomy. Unfortunately, in
a rural area  like my home of mid-coast Maine,  most educators through
high school believe  in a more classically  “practical” application of
knowledge. This  meant that my  dreams of unraveling the  mysteries of
the Cosmos were abandoned amongst rhetoric that there was no future in
being an astronomer, and instead to apply my skill with mathematics to
a more marketable career like engineering.  With this in mind, my core
studies while pursuing my Bachelor’s degree at Bowdoin College did not
include  physics  or  astronomy,  but  instead  mathematics,  computer
science, economics, and German. In the eleventh hour of my senior year
at Bowdoin,  I made a  simple, yet  life-altering decision: I  took an
internship as  a geospatial analyst  in the Earth science  division at
NASA Langley Research Center and abandoned what was shaping up to be a
career as a consultant.  Being  an effective NSF Fellow means stepping
boldly into the  unknown regions of your chosen field,  where you will
have by  definition little  to no  prior knowledge  to work  with, and
nevertheless  making significant  contributions and  becoming a  world
expert.   \textbf{This  ability  to  start from  scratch  and  quickly  become
accomplished in a research topic  is one that I demonstrate repeatedly
throughout  the career  I  detail below,  and one  that  I will  apply
relentlessly if honored with an NSF Fellowship.}


\underline{\textit{Relevant Experience:}}
My  research topic  as part  of a  team in  the NASA  National DEVELOP
Program was analyzing air quality in Houston, TX and correlating MODIS
satellite data with  new NASA High Spectral  Resolution Lidar (HSRL-2)
measurements  to determine  the effectiveness  of HSRL-2  at assessing
small  ($\textless2.5\mu   m$)  particulate   matter  levels   at  the
ground. Despite having no knowledge of remote sensing and only my math
and computer science  skills, I quickly became  proficient with Matlab
and was  invaluable as the data  and statistical analyst on  the team.
At the end of the project, the  team and center leads requested that I
travel  to NASA  Headquarters due  to  my expertise,  and present  the
results of  the analysis in  both a talk and  a poster, which  gave me
valuable presentation experience.

After completing my research project  at NASA Langley, speaking to numerous
scientists, and gaining the experience of  working at a NASA center, I
finally  understood  that  a  career in  space  science  research  was
plausible and  exactly what  I wanted  with my  future.  With  this in
mind, I  moved to NASA's  Goddard Space  Flight Center and  became the
research team lead for a  programming intensive Earth science research
project.   Once  again,  despite  my   lack  of  experience  with  the
programming language  Python, I  learned quickly  and became  the head
developer for a remote-sensing  powered, automated landslide detection
software package  for Nepal written  entirely in Python.   During this
project, in  addition to gaining additional  experience presenting the
results of  my research at  NASA headquarters  for the second  time, I
learned  to  both  lead  and   work  effectively  within  a  group  of
researchers.  Furthermore,  the automated landslide  detection product
was handed  off directly to  end-users in  Nepal, and as  the research
team lead  I was  the main  contact tasked  with ensuring  the product
satisfied the needs of the end-users. The culmination of this position
at NASA Goddard  included making the  product open-source in hopes  that it
could be  adjusted to other  regions of  the world, and  a publication
that is currently under review in \textit{Earth Interactions}.

During the same  time period, I also worked for  NASA's Applied Remote
Sensing  Education  and  Training   (ARSET)  program,  where  I  wrote
research-oriented python programs and  gained experience training NGOs
and government  organizations in the  use of NASA software  and remote
sensing  data.    Meanwhile,  I  attended  every   planetary  science,
astronomy, astrophysics, and heliophysics talk I could and spoke to as
many scientists  as possible in  an attempt to pinpoint  my interests,
and  possibly attain  a research  assistant position  in one  of those
divisions. Eventually I  met a planetary scientist,  Dr.  Conor Nixon,
who   was   working  on   the   Cassini   CIRS  instrument   team   at
Goddard. Despite  my complete inexperience with  planetary science and
atmospheric modeling, Dr. Nixon recognized my passion, motivation, and
intelligence and  hired me as  a planetary science  research assistant
for the CIRS team.

After deciding at the end of my  first NASA internship that I would to
pursue a future in space science, \textbf{it  had taken me less than a year to
go from an  Earth science intern with little research  experience to a
fully-fledged research assistant in  the planetary science division at
NASA Goddard.}  Again  I found  myself in  a position  requiring programming
languages that I had no  experience with, and knowledge of atmospheric
modeling  and infrared  spectroscopy that  I lacked.   Nevertheless, I
committed myself to becoming an expert in my research topic: using the
Cassini  CIRS   far-infrared  interferometer  to  determine   the  D/H
abundance on  Saturn and  Jupiter. By asking  endless questions  of my
advisor, reading  a litany of  papers for background  information, and
immersing myself in  the atmospheric modeling software,  I swiftly was
up  to   speed  and  contributing   to  the  CIRS  research   team  at
Goddard. Over the next year, I  worked and problem solved primarily on
my own, but with overall guidance from my advisor and other members of
the CIRS team.  I gained  invaluable experience working through issues
in my research  independently, as well as added Fortran  and IDL to my
list of  proficient programming  languages.  \textbf{As the  lead author  on a
publication in the  Astronomical Journal with 9  other co-authors} that
resulted  from  this work  \citep{Pierel:2017},  I  became skilled  at
technical writing, compromising with  co-authors around the world, and
the publication process as a whole.  This measurement of D/H had never
been made with the high spectral resolution CIRS data, and the results
showed definitively for the first time that the deuterium abundance on
Saturn is  lower than  that of  Jupiter. This  is contrary  to current
model predictions, which suggests  that our understanding of planetary
formation and  evolution is incomplete,  a result that will  spur much
research in the future to discover the cause of this discrepancy.

I  knew once  I  had decided  that  I  wanted to  pursue  a career  in
astronomy that I would need to obtain  a PhD if I wanted to conduct my
own  research  and  return  to  NASA as  a  fully  qualified  research
scientist.  During my  year as a research assistant for  the CIRS team
at  Goddard,  I  was  accepted  to the  Physics  PhD  program  in  the
Department of Physics $\&$ Astronomy  under advisor Dr.  Steven Rodney
at  the  University  of  South  Carolina, where  I  would  finally  be
conducting astronomical research.  With no previous physics education,
I was  accepted with the  understanding that  I would need  to quickly
prove  that this  would  not hinder  me academically  in  any way.   I
completed  my first  year of  graduate education  with a  3.9 GPA  and
passed  my  qualification  exams  on  the  first  of  three  attempts,
solidifying  my position  as  a PhD  candidate. \textbf{As a PI  I was  awarded
a \url{~}$\$$50,000  NASA HST Archival Research  Grant titled ``Turning
Gravitational  Lenses into  Cosmological Probes'',  and as a Co-I I  was
awarded  a $\$$25,000  NASA  EPSCoR grant  titled  ``Rare and  Peculiar
Stellar    Explosions   with    the   Next    Generation   of    Space
Telescopes''.} Under the second  grant, I already have  a publication in
preparation as lead author in which I have extrapolated supernova (SN)
spectral energy distribution (SED) templates into the IR and UV, to be
used with next generation space telescopes JWST and WFIRST.  Under the
first grant,  I will  be developing  a standardized  software package
capable of  measuring time delays  and lensing properties  of strongly
gravitationally lensed and multiply imaged SNe.

\underline{\textit{Future Goals: }}
My first goal, which I am already well on my way toward accomplishing,
is to obtain  my PhD from the Department of  Physics $\&$ Astronomy at
the University of South Carolina, which  will make me the first member
of my family  to obtain such a  degree. I will tailor  my research and
the software  I develop during this  time to be relevant  for the next
generation  of  space  telescopes,  namely  JWST  and  WFIRST.   After
completing my  PhD, I  plan to  return to NASA  and become  a research
scientist  analyzing data  from  one of  these  telescopes, where  the
expertise  and global  connections  gained through  an NSF  Fellowship
would make me a valuable and effective researcher.

\underline{\textit{Summary of Intellectual Merit: }} Time and time again I’ve proven my ability to master a subject effectively and efficiently. I continue to challenge myself by stepping into an area of research I have no experience with, and nevertheless have been extremely successful. After starting my quest to become an astronomer in 2014 with no experience, I have already gained a complex understanding of remote sensing, statistical analysis, atmospheric modeling, technical writing, STEM outreach, SN classification, gravitational lensing, and much more.I am proficient with Matlab and R, I can program in seven languages,
and I have shown through my success thus far in my Physics degree that
the obstacle  of stepping into  the unkown is  no obstacle at  all. In
short, I have a plethora  of research experiences and technical skills
that, in addition my aptitude  for mastering new subjects quickly, set
me up to become a world expert  in analyzing multiply imaged SNe and a
successful NSF Fellow.



